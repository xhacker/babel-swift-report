\documentclass{sfuthesis}

\title{Babel Swift: An Objective-C to Swift Conversion Tool}
\author{Dongyuan Liu}
\degree{Bachelor of Science}
\discipline{Computing Science}
\department{School of Computing Science}
\faculty{Faculty of Applied Science}
\copyrightyear{2016}
\semester{Spring 2016}
%\date{1 April 2016}
\copyrightnotice{This work is licensed under the Creative Commons
Attribution-NonCommercial-ShareAlike 4.0 International
(\url{http://creativecommons.org/licenses/by-nc-sa/4.0/})}

\keywords{Objective-C, Swift, Source Code Transformation}

%\committee{%
%	\chair{Pamela Isely}{Professor}
%	\member{Emmett Brown}{Senior Supervisor\\Professor}
%	\member{Bonnibel Bubblegum}{Supervisor\\Associate Professor}
%	\member{James Moriarty}{Supervisor\\Adjunct Professor}
%	\member{Kaylee Frye}{Internal Examiner\\Assistant Professor\\School of Engineering Science}
%	\member{Hubert J.\ Farnsworth}{External Examiner\\Professor\\Department of Quantum Fields\\Mars University}
%}

%   PACKAGES AND CUSTOMIZATIONS  %%%%%%%%%%%%%%%%%%%%%%%%%%%%%%%%%%%%%%%%%%%%%%
%
%   Add any packages or custom commands you need for your thesis here.
%   You don't need to call the following packages, which are already called in
%   the sfuthesis class file:
%
%   - appendix
%   - etoolbox
%   - fontenc
%   - geometry
%   - lmodern
%   - nowidow
%   - setspace
%   - tocloft
%
%   If you call one of the above packages (or one of their dependencies) with
%   options, you may get a "Option clash" LaTeX error. If you get this error,
%   you can fix it by removing your copy of \usepackage and passing the options
%   you need by adding
%
%       \PassOptionsToPackage{<options>}{<package>}
%
%   before \documentclass{sfuthesis}.
%

\usepackage{amsmath,amssymb,amsthm}
\usepackage[pdfborder={0 0 0}]{hyperref}
\usepackage{graphicx}
\usepackage{caption}
\usepackage[cache=false,outputdir=.texpadtmp]{minted}






%   FRONTMATTER  %%%%%%%%%%%%%%%%%%%%%%%%%%%%%%%%%%%%%%%%%%%%%%%%%%%%%%%%%%%%%%
%
%   Title page, committee page, copyright declaration, abstract,
%   dedication, acknowledgements, table of contents, etc.
%

\begin{document}

\frontmatter
\maketitle{}
\makecommittee{}

\begin{abstract}
Swift is a new programming language developed by Apple, designed to replace Objective-C as the main programming language on Apple platforms. This report introduces Babel Swift, a new conversion tool for converting Objective-C code to Swift. Our tool focuses on converting code snippets. The most important feature of Babel Swift is the ability to convert incomplete code snippets with undefined identifiers. Furthermore, Babel Swift could generate Swift-only syntax. In the report, we describe the design and implementation of Babel Swift. We also evaluate Babel Swift by comparing with other existing tools.
\end{abstract}


%\begin{dedication} % optional
%\end{dedication}


%\begin{acknowledgements} % optional
%Foremost, I would like thank my supervisor Dr. Thomas Shermer for his  earnest support.
%\end{acknowledgements}

\addtoToC{Table of Contents}\tableofcontents\clearpage
\addtoToC{List of Tables}\listoftables\clearpage
\addtoToC{List of Figures}\listoffigures\clearpage
\addtoToC{List of Listings}\listoflistings





%   MAIN MATTER  %%%%%%%%%%%%%%%%%%%%%%%%%%%%%%%%%%%%%%%%%%%%%%%%%%%%%%%%%%%%%%
%
%   Start writing your thesis --- or start \include ing chapters --- here.
%

\mainmatter%

\chapter{Introduction}

\section{Objective-C and Swift}

Objective-C is a programming language which adds Smalltalk's object-oriented features to the C programming language, designed by Brad Cox and Tom Love in 1983 \cite{cox1983object}. Objective-C allows programmers to use all the features in the C programming language, together with features of object-oriented paradigm (classes, objects, instance and class methods, etc.). It is the primary programming language for writing software on Apple's OS X and iOS platforms \cite{aboutobjc}.

In 2014, Apple introduced Swift, a new programming language for building iOS, OS X, watchOS, and tvOS apps. Swift adopts safe programming patterns and modern features, makes writing programs easier and more flexible \cite{aboutswift}. Some additional features of Swift include tuples, generics, functional programming features, built-in error handling, and advanced control flow (\texttt{do}, \texttt{guard}, \texttt{defer}, and \texttt{repeat} keywords). It also has a new keyword \texttt{let} to define constants, and a new feature known as \emph{optionals} to ensure safety. Because of the its advanced design, Swift is becoming more and more popular. According to a developer survey by Stack Overflow, a popular question-and-answer website for developers, in 2016, Swift is the third trending technology \cite{stackoverflow2016survey}. However, there are a lot of existing source code and resources in Objective-C. On GitHub, a popular website for hosting source code, by April 2016, there are 36,568 active Objective-C repositories, which is more than 3 times as many as 11,138 active Swift repositories \cite{githut}.

\section{Objective-C to Swift Conversion Tools}

In a Swift project, programmers may want to use existing Objective-C code. Although Objective-C and Swift files can coexist in a single project \cite{swiftandobjc}, it is not possible to use Objective-C directly in a Swift source file. To overcome this limitation, programmers usually convert Objective-C code to Swift manually. However, this manual process is not only time consuming, but also error-prone. Objective-C to Swift conversion tools provide a safe and convenient solution to this problem. 

There are several existing conversion tools: objc2swift \cite{objc2swift} is an open-source conversion tool. Swiftify \cite{swiftify} and iSwift \cite{iswift} are two commercial conversion tools.

The limitation of the existing conversion tools is that they are all designed for converting complete source files or complete projects. They have very limited support for incomplete code snippets with undefined identifiers. However, converting incomplete code snippets is a typical use case of conversion tools. For example, a programmer may find an Objective-C code snippet in a Stack Overflow answer, and wants to use it in a Swift project.

\section{Introducing Babel Swift}

This report introduces Babel Swift, a new conversion tool for converting Objective-C code to Swift. Babel Swift is designed for handling incomplete code snippets with undefined identifiers. It could also generate Swift-specific code, e.g. it has the ability to convert C-style \texttt{for} loops to \texttt{for-in} statements and \texttt{stride} in Swift. Babel Swift is written in Python, based on libclang \cite{libclang}. The full source code could be found on GitHub \cite{babelswift}.

\section{Organization of this Report}

The rest of this report is organized as follows. We describe the objectives and implementation in Chapter 2. We briefly discuss the alternative approaches in Chapter 3. In Chapter 4, we evaluate our conversion tool, and compare it with other existing tools. In Chapter 5, we conclude the report.

\chapter{Objectives and Implementation}

Babel Swift is an Objective-C to Swift code conversion tool. The main objective is to convert incomplete code snippets.

\section{Overall Design}

Babel Swift is written in Python, based on libclang \cite{libclang} and its Python binding. libclang is a C interface to Clang \cite{clang}, a C language family frontend for LLVM. libclang provides a small API for parsing source code and accessing the parsed abstract syntax tree (AST). By using Clang as the parser, we can focus on improving the conversion itself.

In libclang, a \emph{cursor} represents a location within the AST. Our transform function takes the root cursor as input, and outputs the converted Swift code. We use a depth-first traversal to recursively visit all the nodes in the AST. We will discuss how we generate the AST, and how we generate the Swift code from the AST in the following sections.

\section{Handling Code Snippets}

In this report, we use the term \emph{code snippet} to represent a piece of code \emph{within} a function. A code snippet consists of one or more statements. A code snippet could not contain any function definitions. Listing \ref{lst:snippet} is an example of Objective-C code snippet (\texttt{NSPopover} is a class in \emph{AppKit}, a foundational framework used in Mac development).

\begin{listing}
\caption{A code snippet}
\label{lst:snippet}
\begin{minted}{objc}
NSInteger answer = 42;
if (answer == 42) {
    NSPopover *popover = [[NSPopover alloc] init];
    [popover run];
}
\end{minted}
\end{listing}

Although it seems to be easy to handle code snippets, libclang refuses to parse them. For Objective-C, libclang expects the input to be a complete source file. To overcome this limitation, Babel Swift wraps the input code snippet into a complete source file. In the example above, the wrapped input for libclang is:

\begin{listing}
\caption{A complete source file containing the snippet in Listing \ref{lst:snippet}}	
\label{lst:wrappedsnippet}
\begin{minted}{objc}
#import <AppKit/AppKit.h>
#import "BabelSwiftHeader.h"

@implementation __BABEL_SWIFT_WRAPPER_CLASS__

- (id)__BABEL_SWIFT_WRAPPER_METHOD__ {

NSInteger answer = 42;
if (answer == 42) {
    NSPopover *popover = [[NSPopover alloc] init];
    [popover run];
}

}

@end
\end{minted}
\end{listing}

We use two special names \texttt{\_\_BABEL\_SWIFT\_WRAPPER\_CLASS\_\_} and \texttt{\_\_BABEL\_\-SWIFT\_\-WRAPPER\_\-METHOD\_\_} to wrap the snippet. \texttt{@implementation \_\_BABEL\_SWIFT\_\-WRAPPER\_\-CLASS\_\_} and \texttt{@end} define a class, \texttt{- (id)\_\_BABEL\_SWIFT\_WRAPPER\_METHOD\_\_} defines a method. We also import two headers: \texttt{AppKit/AppKit.h} contains the fundamental declarations in Cocoa\footnote{Apple's native object-oriented API for its OS X operating system}, e.g. \texttt{NSInteger} and \texttt{NSDictionary}. \texttt{BabelSwiftHeader.h} is for handling \emph{incomplete code snippets}, which will be elaborated in the next section.

% TODO: show AST and how to convert them

\section{Handling Incomplete Code Snippets}

We define an \emph{incomplete code snippet} as a code snippet which contains undefined identifiers. For example, the following Objective-C code snippet is an \emph{incomplete code snippet}, because both \texttt{speed} and \texttt{TimeMachine} are undefined identifiers.

\begin{listing}
\caption{An incomplete code snippet}
\label{lst:incomplete}
\begin{minted}{objc}
if (speed == 88) {
    TimeMachine *timeMachine = [[TimeMachine alloc] init];
    [timeMachine run];
}
\end{minted}
\end{listing}

In Clang, the Sema module is responsible for both semantic analysis and AST construction. Thus, libclang cannot generate the AST for incomplete code snippets because it does not know whether an identifier is a variable name or a class name. For the above snippet, libclang generated the following error messages:

\begin{listing}
\caption{Error messages generated by libclang for snippet in Listing \ref{lst:incomplete}}
\begin{verbatim}
input.m:7:5: error: use of undeclared identifier 'speed'
input.m:8:5: error: use of undeclared identifier 'TimeMachine'
input.m:8:18: error: use of undeclared identifier 'timeMachine'
input.m:8:34: error: use of undeclared identifier 'TimeMachine'
input.m:9:6: error: use of undeclared identifier 'timeMachine'
\end{verbatim}
\end{listing}

As mentioned before, we included a special header file \texttt{BabelSwiftHeader.h}. We can fix the errors by adding pseudo-declarations in this header file. For each undefined variable name, we declare an empty class interface, and declare the variable using that new class. For each undefined class name, we add an empty class interface in the header file. After fixing all the errors, libclang would be able to generate the AST. The header file \texttt{BabelSwiftHeader.h} for Listing \ref{lst:incomplete} is like:

\begin{listing}
\caption{The content of \texttt{BabelSwiftHeader.h} after fixing all the errors}
\begin{minted}{objc}
@class __BABEL_SWIFT_WRAPPER_CLASS__;
@class __BABEL_SWIFT_PSEUDO_CLASS_1__;
@class TimeMachine;

@interface __BABEL_SWIFT_WRAPPER_CLASS__ : NSObject
@end

@interface __BABEL_SWIFT_PSEUDO_CLASS_1__ : NSObject
@end

@interface TimeMachine : NSObject
@end

__BABEL_SWIFT_PSEUDO_CLASS_1__ *speed;
\end{minted}
\end{listing}

In the above listing, the three \texttt{@class}es are forward declarations. The three pairs of \texttt{@interface} and \texttt{@end} are the empty class interfaces.

We use multiple iterations to fix all the errors. For each iteration, we only try to fix the first error reported by libclang. We use the identifier name to guess whether it is a class name or a variable name. If the identifier starts with a capital letter, we assume it is a class name, otherwise we assume it is a variable name. It is obvious to see that we may make wrong guesses, e.g. a programmer could name a variable \texttt{IPAddress}. To make sure all the identifiers are being declared correctly, if it failed to fix the error, we will make another opposite guess. We repeat this approach until there are no errors.

% TODO: chained property access

\section{Generating Swift Code from AST}

After wrapping the snippet into a complete file and fixing all the errors, we use libclang to parse the file and generate the AST. In libclang, each node in the AST is called a \emph{cursor}. Our transform function takes a cursor as input, and outputs the converted Swift code. We use a depth-first traversal to recursively visit all the nodes in the AST.

The AST generated by libclang for Listing \ref{lst:incomplete} would be:

\begin{listing}[H]
\caption{The AST generated by libclang for Listing \ref{lst:incomplete}}
\label{lst:ast}
\begin{verbatim}
 (CursorKind.COMPOUND_STMT)
  +-- (CursorKind.IF_STMT)
     +--== (CursorKind.BINARY_OPERATOR)
     |  +-- (CursorKind.IMPLICIT_CAST_EXPR_STMT)
     |  |  +--speed (CursorKind.DECL_REF_EXPR)
     |  +-- (CursorKind.IMPLICIT_CAST_EXPR_STMT)
     |     +-- (CursorKind.INTEGER_LITERAL)
     +-- (CursorKind.COMPOUND_STMT)
        +-- (CursorKind.DECL_STMT)
        |  +--timeMachine (CursorKind.VAR_DECL)
        |     +--TimeMachine (CursorKind.OBJC_CLASS_REF)
        |     +--init (CursorKind.OBJC_MESSAGE_EXPR)
        |        +--alloc (CursorKind.OBJC_MESSAGE_EXPR)
        |           +--TimeMachine (CursorKind.OBJC_CLASS_REF)
        +--run (CursorKind.OBJC_MESSAGE_EXPR)
           +-- (CursorKind.IMPLICIT_CAST_EXPR_STMT)
              +--timeMachine (CursorKind.DECL_REF_EXPR)
\end{verbatim}
\end{listing}

We use a special notation to describe how the visitor generates code for different cursor types: The left side is the type of the cursor, the right side the the code returned by the visitor. On the right side, \texttt{monospaced} text is the code we are injecting. \textit{italic} text means that it needs to be executed to get the value. $C_{i}$ represents the \emph{i}-th child of the cursor. \emph{visit(c)} represents the code generated by visiting cursor \emph{c}. \emph{this} is the current cursor we are visiting, it has some properties provided by libclang. 

\newcommand{\visitchild}[1]{\textit{visit($C_{#1}$)}}

\begin{table}[H]
\begin{center}
\begin{tabular}{|l|l|l|l|l|}
\hline
\textbf{Cursor Type} & \textbf{Generated Code} \\
\hline

\texttt{DeclRefExpr}, \texttt{ObjCClassRef} & \textit{this.spelling} \\

\texttt{ImplicitCastExprStmt} & \visitchild{1} \\

\texttt{ObjCBoxedExprStmt} & \visitchild{1} \\

\texttt{IntegerLiteral}, \texttt{FloatingLiteral} & \textit{this.spelling} \\

\texttt{ObjCSelfExpr} & \textit{this.spelling} \\

\texttt{ObjCStringLiteral} & \textit{this.spelling} \\

\texttt{ObjCArrayLiteralStmt} & \texttt{[} \visitchild{1} \texttt{,} \visitchild{2} \texttt{,} ... \texttt{,} \visitchild{n} \texttt{]} \\

\texttt{ObjCDictionaryLiteralStmt} & \texttt{[} \visitchild{1} \texttt{:} \visitchild{2} \texttt{,} ... \texttt{,} \visitchild{n-1} \texttt{:} \visitchild{n} \texttt{]} \\

\texttt{CStyleCastExpr} & \visitchild{0} \texttt{as} \textit{this.cstyleCastTargetType} \\

\texttt{DeclStmt} & \texttt{let} \textit{this.spelling} \texttt{=} \visitchild{1} \\

\texttt{ParenExpr} & \texttt{(} \visitchild{1} \texttt{)} \\

\texttt{IfStmt} & \texttt{if} \visitchild{1} \visitchild{2} \\

\texttt{WhileStmt} & \texttt{while} \visitchild{1} \visitchild{2} \\

\texttt{BinaryOperator} & \visitchild{1} \textit{this.spelling} \visitchild{2} \\

\texttt{UnaryOperator} & \textit{this.spelling} \visitchild{1} \\

\texttt{MemberRefExpr} & \visitchild{2} \texttt{.} \visitchild{1} \\

\texttt{CompoundStmt} & \texttt{\{} \visitchild{1} \visitchild{2} ... \visitchild{n} \texttt{\}} \\

\texttt{CallExpr} & \visitchild{1} \texttt{(} \visitchild{2} \texttt{,} ... \texttt{,} \visitchild{n} \texttt{)} \\

\hline
\end{tabular}
\end{center}
\caption{Translation scheme of the visitor}
\end{table}

After applying the translation scheme, we get the following Swift code from the above AST:

\begin{listing}[H]
\caption{Swift code generated from AST in Listing \ref{lst:ast}}
\begin{minted}{swift}
if speed == 88 {
   let timeMachine = TimeMachine()
   timeMachine.run()
}
\end{minted}
\end{listing}

\section{Special Cases}

The translation scheme for most of the cursor types are not complex and can be described by the notation we used. There are a few special cases need different strategies.

\subsection{Swift Data Types}

Swift has a different set of data types than Objective-C. We use mappings to convert primitive data types. Because it is allowed to use C types in Objective-C, we also have mappings for C types. Once we see an Objective-C or C data type in the mapping, we use the Swift counterpart in the output.

\newcommand{\specialcell}[2][c]{%
  \begin{tabular}[#1]{@{}c@{}}#2\end{tabular}}

\begin{table}[H]
\begin{center}
\begin{tabular}{|l|l|l|}
\hline
{\bf Objective-C \cite{objcdatatypes}} & {\bf Swift \cite{swiftdatatypes}} & \specialcell{{\bf Size in 32/64-bit}\\{\bf runtime (bytes)}} \\
\hline
\texttt{char}       & \texttt{Int8}   & 1 / 1 \\
\texttt{BOOL}, \texttt{bool} & \texttt{Int8}   & 1 / 1 \\
\texttt{short}      & \texttt{Int16}  & 2 / 2 \\
\texttt{int}        & \texttt{Int32}  & 4 / 4 \\
\texttt{long}       & \texttt{Int}    & 4 / 8 \\
\texttt{long long}  & \texttt{Int64}  & 8 / 8 \\
\texttt{size\_t}    & \texttt{Int}    & 4 / 8 \\
\texttt{time\_t}    & \texttt{Int}    & 4 / 8 \\
\texttt{NSInteger}  & \texttt{Int}    & 4 / 8 \\
\texttt{NSUInteger} & \texttt{Int}    & 4 / 8 \\
\texttt{float}      & \texttt{Float}  & 4 / 4 \\
\texttt{double}     & \texttt{Double} & 8 / 8 \\
\hline
\end{tabular}
\end{center}
\caption{Mappings for primitive data types}
\end{table}

\subsection{Method Calls}

In Objective-C, the syntax for method calls is a little special compared to other C-like languages. For example, the following code would call \texttt{- insertSubview:atIndex:} with parameters \texttt{mySubview} and \texttt{2}.

\begin{minted}{objc}
[view insertSubview:mySubview atIndex:2];
\end{minted}

Swift uses dot syntax for method calls. The counterpart in Swift is much more like in other languages \cite{swiftobjcapis}:

\begin{minted}{swift}
view.insertSubview(mySubview, atIndex: 2)
\end{minted}

Babel Swift could recognize method calls in Objective-C, extract the method signatures and arguments, and convert them to Swift method calls.

\subsection{Initializations}

In Objective-C, class initializers begin with \texttt{init}, or \texttt{initWith:} if it takes arguments. Programmers also need to call \texttt{alloc} explicitly when initializing an object. A complete object initialization in Objective-C is like:

\begin{minted}{objc}
UITableView *tableView = [[UITableView alloc] initWithFrame:CGRectZero
    style:UITableViewStyleGrouped];
\end{minted}

In Swift, the \texttt{init} or \texttt{initWith:} is removed. \texttt{alloc} does not need to be called anymore. The object initialization becomes:

\begin{minted}{swift}
let tableView: UITableView = UITableView(frame: CGRectZero, style: .Grouped)
\end{minted}

\subsection{\texttt{for} Loops}

\texttt{for} loops are widely used in Objective-C. A typical \texttt{for} loop in Objective-C is like:

\begin{listing}[H]
\caption{A typical Objective-C \texttt{for} loop}
\label{lst:forloop}
\begin{minted}{objc}
for (int index = 0; index < array.count; index++) {
    NSLog(@"%@", array[index]);
}
\end{minted}
\end{listing}

However, C-style \texttt{for} loops are deprecated and will be removed completely in Swift 3.0 \cite{removecforloops}. To follow the best practice in Swift, Babel Swift would generate the following Swift code for the above \texttt{for} loop:

\begin{listing}[H]
\caption{Swift code converted from Listing \ref{lst:forloop}}
\begin{minted}{swift}
for index in 0..<array.count {
    NSLog("%@", array[index])
}
\end{minted}
\end{listing}

A even better way for this case is to eliminate \texttt{index} completely. However, it requires we looking into the \texttt{for} body. This introduces a lot more complexity and we decided not to do it.

%TODO: type inference

%TODO: \section{Improving libclang}
%
%Although libclang provides a set of methods for accessing the AST, it's implementation is incomplete.
%
%unexposed
%
%information in cursor

% TODO: Using PCH for better performance

\section{Remaining Issues}

%++ --

Comments are useful in source code. It would be good to keep the comments during the conversion. Currently, they are simply dropped due to the limitation of libclang. This could be implemented in the future by adding the functionality to keep the comments in libclang.

Another potential improvement is to support larger forms of code like functions and classes in addition to code snippets. It could use the same approach to allow undefined identifiers in functions and classes.

\chapter{Alternative Approaches}

In this chapter, we briefly discuss some alternative approaches to implement the conversion tool. Instead of using libclang, we can also build the tool directly upon Clang (or with LibTooling), or write the parser from scratch.

Babel Swift is based on libclang. As noted before, libclang offers a convenient and stable API for accessing the AST generated by Clang. It also has an official Python binding, which is very helpful for fast prototyping. However, libclang does not provide the full access over the AST. To implement a decent conversion tool, many new featured have to be added in libclang. Moreover, as a young addition to Clang, libclang does not have complete documentation.

Because Apple is the main user and driver behind Objective-C, Clang is the de facto standard frontend for the language. Using a Clang-based approach (including libclang and LibTooling) could save a lot of effort on parsing the Objective-C code.

All the approaches differ in the frontend. Once the code in source language is converted to AST, the implementations of the backend are almost the same.

\section{Building Directly Upon Clang}

Building the tool directly upon Clang provides the full control over the AST. There are two major drawbacks: First, Clang itself is not designed to be used as a library. Setting up Clang to use it as a library is extremely complex. Second, Clang is evolving rapidly. It does not promise a stable API. If the tool is building upon a certain version of Clang, there is a good chance that the code will broken with the next version. In addition, the only possible language to implement based on Clang is C++.

\subsection{LibTooling}

LibTooling is a library for writing standalone tools based on Clang \cite{libtooling}. As noted before, Clang has a complex setup process. Instead, LibTooling offers a straightforward setup process. LibTooling is an ideal framework to develop tools based on Clang.

Despite the drawbacks (unstable API, cannot use languages other than C++), if the objective is to build a full-fledged conversion tool, LibTooling is a better choice than libclang because of its full control over the AST.

\section{Custom Parser}

Using a custom parser provides the best flexibility over the implementation. The parser could be written by hand or generated by parser generators. The most obvious drawback is that there are a lot of effort to implement a complete parser. objc2swift \cite{objc2swift} is using this approach.

\chapter{Evaluation}

Evaluating a code conversion tool is difficult, because there are no obvious metrics for them. Typical metrics like performance are not important for code conversion tools. Instead, the most important qualities for code conversion tools are \emph{the support for input source code}, \emph{the correctness of conversion}, and \emph{the quality of converted code}. This chapter defines these terms and attempts to evaluate these qualities in the following sections.

\section{Methodology}

To collect real world Objective-C code for evaluation, we chose most voted questions under \textbf{objective-c} tag on Stack Overflow. If the question and its answers contain code, we can use them for our evaluation. We have collected 6 pieces of code in total: \emph{dispatch.m}, \emph{location.m}, \emph{prepareforsegue.m}, \emph{sortusingblock.m}, \emph{textfield.m}, and \emph{viewwillappear.m}.

Because Babel Swift only supports converting code snippets, we did the two steps to preprocess the code:

\begin{enumerate}
  \item If the code contains more than one functions, we only keep the first function. We also remove dangling code other than that function.
  \item If the code contains a function, we use the function body as the input for Babel Swift; we use the whole function as the input for other conversion tools.
\end{enumerate}

After preprocessing, we run Babel Swift, objc2swift, Swiftify, and iSwift on all the 6 inputs.

\section{The Support of Input Source Code}

If a conversion tool succeeded in converting a piece of source code, we say that it \emph{supports} that piece of code. Table \ref{table:numcodesupported} shows whether a conversion tool support a certain piece of code.

\begin{table}[H]
\begin{center}
\begin{tabular}{|l|l|l|l|l|}
\hline
\textbf{Code Piece} & Babel Swift & objc2swift & Swiftify & iSwift \\
\hline
dispatch        & Yes & No  & Yes & Yes \\
location        & Yes & No  & Yes & No  \\
prepareforsegue & Yes & No  & Yes & No  \\
sortusingblock  & Yes & No  & Yes & No  \\
textfield       & No  & No  & Yes & No  \\
viewwillappear  & Yes & No  & Yes & No  \\
\hline
\textbf{Total}  & 5 (83.3\%) & 0 (0\%) & 6 (100\%) & 1 (16.7\%) \\
\hline
\end{tabular}
\end{center}
\caption{The support of input source code}
\label{table:numcodesupported}
\end{table}

\section{The Correctness of Conversions}

If the converted code and the original code have the same functionality, we say that the conversion is \emph{correct}. It is hard to write a general purpose tool to test the code automatically. So we evaluated the correctness of conversions by hand. Table \ref{table:correctness} shows the correctness of conversions for each code piece.

\begin{table}[H]
\begin{center}
\begin{tabular}{|l|l|l|l|l|}
\hline
\textbf{Code Piece} & Babel Swift & objc2swift & Swiftify & iSwift \\
\hline
dispatch        & a & n/a & c & d \\
location        & a & n/a & c & n/a \\
prepareforsegue & a & n/a & c & n/a \\
sortusingblock  & a & n/a & c & n/a \\
textfield       & a & n/a & c & n/a \\
viewwillappear  & a & n/a & c & n/a \\
\hline
\end{tabular}
\end{center}
\caption{The correctness of conversions}
\label{table:correctness}
\end{table}

\section{The Quality of Converted Code}

Finally, we convert all the test cases by hand. If the machine-converted code is \emph{identical} to the hand-converted one, we say that the conversion is \emph{optimal}. By \emph{identical}, we do tolerate extra or lacking of spaces. Table \ref{table:quality} shows the quality of converted code for each conversion tool.

\begin{table}[H]
\begin{center}
\begin{tabular}{|l|l|l|l|l|}
\hline
\textbf{Code Piece} & Babel Swift & objc2swift & Swiftify & iSwift \\
\hline
dispatch        & a & n/a & c & d \\
location        & a & n/a & c & n/a \\
prepareforsegue & a & n/a & c & n/a \\
sortusingblock  & a & n/a & c & n/a \\
textfield       & a & n/a & c & n/a \\
viewwillappear  & a & n/a & c & n/a \\
\hline
\end{tabular}
\end{center}
\caption{The quality of converted code}
\label{table:quality}
\end{table}

\section{Evaluation Summary}

Waiting for implementation to be finalized.

\chapter{Conclusion}

This report presents the design and implementation of a new Objective-C to Swift conversion tool. The key features of our tool include the ability to convert incomplete code snippets, as well as generating Swift-specific code. In the report, we describe the techniques we used to achieve the objectives. We also briefly discuss the alternative approaches. In the future, this tool could be improved to support comments. Another improvement is to support converting larger forms of code including functions and classes.
One key contribution of our work is the implementation for the design. The implementation of Babel Swift is complete and useable. The source code is fully accessible on GitHub \cite{babelswift}. At last, we evaluate the conversion tool. Results show that this is the best tool for converting code snippets among the tools we evaluated.


%   BACK MATTER  %%%%%%%%%%%%%%%%%%%%%%%%%%%%%%%%%%%%%%%%%%%%%%%%%%%%%%%%%%%%%%
%
%   References and appendices. Appendices come after the bibliography and
%   should be in the order that they are referred to in the text.
%
%   If you include figures, etc. in an appendix, be sure to use
%
%       \caption[]{...}
%
%   to make sure they are not listed in the List of Figures.
%

\backmatter%
	\addtoToC{Bibliography}
	\bibliographystyle{plain}
	\bibliography{references}

\begin{appendices} % optional
	\chapter{Code}
\end{appendices}
\end{document}
